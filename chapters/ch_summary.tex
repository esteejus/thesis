\chapter{Summary}
The need for precise experimental pion data is paramount to understanding exotic, dense matter, such as neutron stars. Besides direct observation of these objects, the lab is the best way to probe the density dependence of the symmetry energy, and is the only way to probe different asymmetries. Pions have been proposed to be sensitive to the high density regions of these collisions, making it an ideal probe to constrain the symmetry energy. Though simulating pion production has proved to be quite challenging, it has motivated the design of high efficiency detectors such as the \spirit TPC to measure pion emission from neutron rich heavy ion collisions. 
We preformed a campaign of experiments measuring collisions of neutron rich, radioactive beams, on stable targets in 4 different configurations -- $\tin{132}{124}, \tin{108}{112}, \tin{112}{124},$ and $\tin{124}{112}$. All measured with the \spirit TPC which was able to measure pions to very low energies without the need for any extrapolations and very little assumptions. 

\subsection{Results}
The pion yield, ratio, and some new observables -- namely the double ratio-- were measured to within <4\% accuracy. We observed a deviation from the  naive $\pi^-/\pi^+ ~ (N/Z)^2$ scaling expected from the $\Delta$-resonance model discussed in Section~\ref{sec:pionObs}. 




\subsection{Outlook}


