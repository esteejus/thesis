\chapter{Summary}
The need for precise experimental pion data is paramount to understanding exotic, dense matter, such as neutron stars. Besides direct observation of these objects, the lab is the best way to probe the density dependence of the symmetry energy, and is the only way to probe different asymmetries. Pions have been proposed to be sensitive to the high density regions of these collisions, making it an ideal probe to constrain the symmetry energy. Though simulating pion production has proved to be quite challenging, it has motivated the design of high efficiency detectors such as the \spirit TPC to measure pion emission from neutron rich heavy ion collisions. We have preformed a campaign of experiments measuring collisions of neutron rich, radioactive beams, on stable targets in 4 different configurations -- $\tin{132}{124}, \tin{108}{112}, \tin{112}{124},$ and $\tin{124}{112}$, all measured with the \spirit TPC. 

\subsection{Results}
The pion yield, ratio, and some new observables -- namely the double ratio-- were measured to within <4\% accuracy. Marking the first time pions have been measured in the sub-threshold region. We observed a deviation from the  naive expectation in a simple $\Delta$ resonance model, where $\pi^-/\pi^+ ~ (N/Z)^2$, as discussed in Section~\ref{sec:pionObs}. This is somewhat expected, as the mean field potential contributes to the chemical potential of pions in a thermal statistical equilibrium simple model. Although it is clear, that other effects such as the relatively unknown isoscalar and isovector $\Delta$ potential plays an important role. The pion spectral ratio was also measured with a high degree of accuracy, also providing much needed information about the dynamics of pions, even to very low pion energies, where previous experiments failed to measure. This is a promissing observable since high energy pions are more likely to exit the nuclear matter sooner and unaffected by re-scattering and adsorption-re-emission processes, which dilute the observable. In this way high energy pions may be more sensitive to the earlier high density timescales of the collision, whereas the low energy pions suffer from other influences, namely the Coulomb potential, pion optical potentials, and the $\Delta$ potential. 

\subsection{Outlook}
Theoretical collaboration, which has made considerable progress improving details of the calculation, will further improve in light of this new data. It is clear that a serious effort must be made to include effects such as the symmetry potentials of pions and deltas. Which most codes do not. It also appears to be possible to make first constraints on the generally unknown $\Delta$ potential by fitting the pion yield data \cite{cozmaPC}. Making such a constraint would allow for the possibility to start making constraints on the density dependence of the symmetry energy. 

