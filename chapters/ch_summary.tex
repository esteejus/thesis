\chapter{Summary}
\label{chap:summary}

The need for precise experimental constraints on the symmetry energy at high densities is paramount to understanding exotic, dense matter, such as neutron stars. Direct observation of these objects provides information about the total equation of state, which is primarily sensitive to the pressure in neutron matter. Only by laboratory measurements can one control the asymmetry of matter and isolate its dependence on the isospin asymmetry $\delta = \frac{\rho_n - \rho_p}{\rho}$. This information is critical to understanding the symmetry energy and associated isovector mean field potentials. These mean field potential contribute strongly to the chemical potentials, which control the phase transitions and the internal structure within stars. Heavy ion collisions provide the only laboratory environment to study these issues at the suprasaturation densities relevant to the neutron star interior.  

Pions have been proposed to be sensitive to the high density regions of these collisions, making it a compelling probe to constrain the symmetry energy. Moreover, questions about what are the stable phases at twice saturation densities cannot be answered without understanding the mean field potentials for pions and deltas in such matter. Though simulating pion production has its challenges, the prospect of answering such compelling questions has motivated the design of high efficiency detectors such as the \spirit TPC to measure pion emission from neutron rich heavy ion collisions. We have preformed a campaign of experiments measuring collisions of neutron rich, radioactive beams, on stable targets in 4 different configurations -- $\tin{132}{124}, \tin{108}{112}, \tin{112}{124},$ and $\tin{124}{112}$, all measured with the \spirit TPC. 

\subsection{Results}
The pion yield, ratio, and some new observables -- namely the double ratio-- were measured to within <4\% accuracy. We find that a major fraction of the pions are emitted below the thresholds of the FOPI experiment, demonstrating that the systematic errors are of the pion ratios from that previous experiments are completely dominated by the extrapolation of the FOPI data to the emission threshold. Based on this information, it is clear that prior conclusions based on comparison of those extrapolated data to theoretical calculations of total pion yields do not take the systematic errors of that extrapolation into account and any agreement their conclusions would contain error but to an unknown degree. 

  This marks the first time pions have been accurately measured in the sub-threshold region. We observed significant a deviation from the  n\"aive expectation of the simple $\Delta$ resonance model, where $\pi^-/\pi^+ ~ (N/Z)^2$, as discussed in Section~\ref{sec:pionObs}. Such a deviation is also predicted by theory, so this explanation should be retired in favor of something more quantitative and informative. Obtaining a better explanation from current calculations is difficult because nearly all pion production calculations are incomplete and each model has a mixture of strengths and inadequacies that are unique to each model. Thus, it is difficult to get the theorists to agree about the consequences of some of the model assumptions that they employ; and consequently a constraint on the density dependence of the symmetry energy. 
  
  %With this wealth of new data that we have obtained, we now begin to work with theorists to understand what new experimental pion observable is sensitive to which theoretical assumption. We will are exploring how the isoscalar and isovector $\Delta$ potential plays an important role in the pion production spectra and emission rates. 
  Also for the first time, the pion spectra was measured. The spectral ratio appears to provide information that is relevant to this question also providing much needed information about the dynamics of pions, even to very low pion energies. Of particular importance is the sensitivity of calculations to the high energy tails of the pion spectra. This is a promising observable since high energy pions  exit the nuclear matter sooner and are less affected by re-scattering and adsorption-re-emission processes, which dilute information about the EOS at the highest density. We expect that high energy pions will be more sensitive to the prevailing conditions at high density, while low energy pions are more sensitive to the energy available for their emission, such as the Coulomb potential, the pion optical potentials, and the $\Delta$ potential. 


\subsection{Outlook}
Theoretical collaboration, which has made considerable progress improving details of the calculation, will further improve in light of this new data. It is clear that a serious effort must be made to include effects such as the symmetry potentials of pions and deltas; which most codes do not. It also appears to be possible to make first constraints on the generally unknown $\Delta$ potential by fitting the pion yield data \cite{cozmaPC}. Making such a constraint would allow for the possibility to start making constraints on the density dependence of the symmetry energy. The interactions with theoretical community that has coalesced around this effort is very exciting, and we are seeing real progress in transport theory. It seems very possible that the theoretical efforts will converge and provide solid interpretations of these and other data much more quickly than they did in the case of the symmetric matter EoS. 

